\documentclass{article}

\usepackage{amsmath}
\usepackage{amssymb}
\usepackage{amsthm}
\usepackage{makecell}
\usepackage{float}

\title{
    Description Logics State Features for Planning (dlplan)
}
\author{
Dominik Drexler \\
Link{\"o}ping University, Link\"oping, Sweden\\
dominik.drexler@liu.se
}

\begin{document}
\maketitle

\newcommand{\emptyelement}[1]{\ensuremath{\mathit{Empty}(#1)}}
\newcommand{\nullaryelement}[1]{\ensuremath{\mathit{Nullary}(#1)}}

\newcommand{\countelement}[1]{\ensuremath{\mathit{Count}(#1)}}
\newcommand{\conceptdistanceelement}[3]{\ensuremath{\mathit{ConceptDistance}(#1, #2, #3)}}
\newcommand{\sumconceptdistanceelement}[3]{\ensuremath{\mathit{SumConceptDistance}(#1, #2, #3)}}
\newcommand{\roledistanceelement}[3]{\ensuremath{\mathit{RoleDistance}(#1, #2, #3)}}
\newcommand{\sumroledistanceelement}[3]{\ensuremath{\mathit{SumRoleDistance}(#1, #2, #3)}}

\section{Introduction}

This document aims at providing a better understanding of the different types of elements
and different elements itself that can be used with this library.
There are four different types of elements: concepts, roles, boolean, and numericals.
The main difference is that each type returns different types of objects when evaluated on a given planning state.
Evaluating a concept returns a set of unary relations over objects,
evaluating a role returns a set of binary relations over objects,
evaluating a boolean returns a Boolean, and
evaluating a numerical returns a natural number.
The following section shows how elements can be composed to obtain more complex elements.

\section{Available Elements}

In this section, we describe the elements that are available to construct more complex elements.
We make use of description logics \cite{baader-et-al-2003} and include Boolean and numerical
elements as described in \cite{drexler-et-al-arxiv2021}.

Consider concepts $C, D$, roles $R, S, T$,
the universe $\Delta$ containing all objects,
and $X$ to be either a concept or a role.
Also consider some predicate $p(c_0,\ldots,c_{n-1})$ with arity $n$,
integers $0\leq i,j\leq n-1$, integer $k\in\{0, 1\}$, and
some constant $a$ that occurs in every instance.

\subsection{Syntax Overview}

\subsubsection{Concepts}

\begin{table}[H]
    \centering
    \begin{tabular}{ccc}
    Name & Abstract syntax & Concrete syntax \\
    \hline
    Primitive concept & $p[i]$ & c\_primitive(p,i) \\
    Top & $\top$ & c\_top \\
    Bottom & $\bot$ & c\_bot \\
    Intersection & $C\sqcap D$ & c\_and(C, D) \\
    Union & $C\sqcup D$ & c\_or(C, D) \\
    Negation & $\neg C$ & c\_not(C) \\
    Difference & $C\setminus D$ & c\_diff(C, D) \\
    Value restriction & $\forall R.C$ & c\_all(R, C) \\
    Existential quantification & $\exists R.C$ & c\_some(R, C) \\
    Role-value-map & $R\subseteq S$ & c\_subset(R, S) \\
     & $R = S$ & c\_equal(R, S) \\
    One-of & $a$ & c\_one\_of(a) \\
    Projection & $R[k]$ & c\_projection(R, k) \\
    \end{tabular}
    \caption{Concepts}
\end{table}

\subsubsection{Roles}

\begin{table}[H]
    \centering
    \begin{tabular}{ccc}
    Name & Abstract syntax & Concrete syntax \\
    \hline
    Primitive role & $p[i,j]$ & r\_primitive(p,i,j) \\
    Universal role & $\top$ & r\_top \\
    Intersection & $R\sqcap S$ & r\_and(R, S) \\
    Union & $R\sqcup S$ & r\_or(R, S) \\
    Complement & $\neg R$ & r\_not(R) \\
    Difference & $R\setminus S$ & r\_diff(R, S) \\
    Inverse & $R^{-1}$ & r\_inverse(R) \\
    Composition & $R\circ S$ & r\_compose(R, S) \\
    Transitive closure & $R^+$ & r\_transitive\_closure(R) \\
    Transitive reflexive closure & $R^*$ & r\_transitive\_reflexive\_closure(R) \\
    Role restriction & $R\vert_{C}$ & r\_restrict(R, C) \\
    Identity & $\mathit{id}(C)$ & r\_identity(C) \\
    \end{tabular}
    \caption{Roles}
\end{table}

\subsubsection{Booleans}

\begin{table}[H]
    \centering
    \begin{tabular}{ccc}
    Name & Abstract syntax & Concrete syntax \\
    \hline
    Empty & \emptyelement{X} & b\_empty(X) \\
    Concept inclusion & $C\sqsubseteq D$ & b\_inclusion(C, D) \\
    Role inclusion & $R\sqsubseteq S$ & b\_inclusion(R, S) \\
    Nullary & \nullaryelement{p} & b\_nullary(p)
    \end{tabular}
    \caption{Booleans}
\end{table}

\subsubsection{Numericals}

\begin{table}[H]
    \centering
    \begin{tabular}{ccc}
    Name & Abstract syntax & Concrete syntax \\
    \hline
    Count & \countelement{X} & n\_count(X) \\
    Concept distance & \conceptdistanceelement{C}{R}{D} & n\_concept\_distance(C, R, D) \\
    Sum concept distance & \sumconceptdistanceelement{C}{R}{D} & n\_sum\_concept\_distance(C, R, D) \\
    Role distance & \roledistanceelement{R}{S}{T} & n\_role\_distance(R, S, T) \\
    Sum role distance & \sumroledistanceelement{R}{S}{T} & n\_sum\_role\_distance(R, S, T) \\
    \end{tabular}
    \caption{Numericals}
\end{table}

\subsection{Semantics}

Our \emph{interpretation} is a state $s$ that consists of ground atoms over a set of predicates.

\subsubsection{Concepts}

\begin{itemize}
    \item $(p[i])^s = \{c_i\in\Delta\mid p(c_0,\ldots,c_i,\ldots,c_{\mathit{arity}(p)})\in s \}$
    \item $\top^s = \Delta$
    \item $\bot^s = \emptyset$
    \item $(C\sqcap D)^s = C^s\cap D^s$
    \item $(C\sqcup D)^s = C^s\cup D^s$
    \item $(\neg C)^s = \Delta\setminus C^s$
    \item $(C\setminus D)^s = (C^s\setminus D^s)$
    \item $(\forall R.C)^s = \{a\mid\forall b:(a,b)\in R^s\rightarrow b\in C^s\}$
    \item $(\exists R.C)^s = \{a\mid\exists b:(a,b)\in R^s\land b\in C^s\}$
    \item $(R\subseteq S)^s = \{a\mid\forall b:(a,b)\in R^s\rightarrow (a,b)\in S^s\}$
    \item $(R = S)^s = \{a\mid\forall b:(a,b)\in R^s\leftrightarrow (a,b)\in S^s\}$
    \item $a^s = \{a\}$
    \item $(R[k])^s = \begin{cases}(\exists R.\top)^s & \text{ if } k = 0\\ (\exists R^{-1}.\top)^s & \text{ if } k = 1 \end{cases}$
\end{itemize}

\subsubsection{Roles}

\begin{itemize}
    \item $(p[i,j])^s = \{(c_i,c_j)\in\Delta\times\Delta\mid p(c_0,\ldots,c_i,\ldots,c_j,\ldots,c_{\mathit{arity}(p)})\in s \}$
    \item $\top^s = \Delta\times\Delta$
    \item $(R\sqcap S)^s = R^s\cap S^s$
    \item $(R\sqcup S)^s = R^s\cup S^s$
    \item $(\neg R)^s = \top^s\setminus R^s$
    \item $(R\setminus S)^s = (R^s\setminus S^s)$
    \item $(R^{-1})^s = \{(b,a)\in\Delta\times\Delta\mid (a,b)\in R^s\}$
    \item $(R\circ S)^s = \{(a,c)\in\Delta\times\Delta\mid (a,b)\in R^s\land (b,c)\in S^s \}$
    \item $(R^+)^s = \bigcup_{n\geq 1}(R^s)^n$
    \item $(R^*)^s = \bigcup_{n\geq 0}(R^s)^n$
    \item $(R\vert_{C})^s = R^s\sqcap (\Delta\times C^s)$
    \item $(\mathit{id}(C))^s = \{(d,d)\mid d\in C^s \}$
\end{itemize}

where the iterated composition $(R^s)^n$ is inductively defined as
$(R^s)^0 = \{(d,d)\mid d\in\Delta\}$ and $(R^s)^{n+1} = (R^s)^n\circ R^s$.

\subsubsection{Booleans}

\begin{itemize}
    \item $\emptyelement{X}^s$ is true iff $|X^s| = 0$
    \item $(C\sqsubseteq D)^s$ is true iff $C^s\subseteq D^s$
    \item $(R\sqsubseteq S)^s$ is true iff $R^s\subseteq S^s$
    \item $\nullaryelement{p}^s$ is true iff $p()\in s$
\end{itemize}

\subsubsection{Numericals}

\begin{itemize}
    \item $\countelement{X}^s\equiv |X^s|$
    \item $\conceptdistanceelement{C}{R}{D}^s$ is the smallest $n\in\mathbb{N}_0$
    s.t.\ there are objects $x_0,\ldots,x_n$ with
    $x_0\in C^s$, $x_n\in D^s$ and $(x_i, x_{i+1})\in R^s$ for $i = 0,\ldots,n-1$.
    Special cases: If $C^s$ is empty then the element evaluates to $0$ and if no such $n$ exists then it evaluates to $\infty$.
    \item $\sumconceptdistanceelement{C}{R}{D}^s:=\sum_{x\in C^s}\conceptdistanceelement{\{x\}}{R}{D}^s$
    where the sum evaluates to $\infty$ if any term is $\infty$.
    \item $\roledistanceelement{R}{S}{T}^s$ is the smallest $n\in\mathbb{N}_0$
    s.t.\ there are objects $x_0,\ldots,x_n$,
    there exists some $(a,x_0)\in R^s$, $(a,x_n)\in T^s$,
    and $(x_i, x_{i+1})\in S^s$ for $i = 0,\ldots,n-1$.
    Special cases: If $R^s$ is emtpy then the element evaluates to $0$ and if no such $n$ exists then it evaluates to $\infty$.
    \item $\sumroledistanceelement{R}{S}{T}^s:=\sum_{r\in R^s}\roledistanceelement{\{r\}}{S}{T}^s$,
    where the sum evaluates to $\infty$ if any term is $\infty$.
\end{itemize}

\section{Feature Generation}

In the feature generation, we place the following additional restrictions on the above elements in order to decrease the combinatorial blowup.
\begin{itemize}
    \item r\_restrict(R, C), r\_inverse(R), r\_transitive\_closure(R), c\_equal(R, S), \\
          r\_transitive\_reflexive\_closure(R) where R, S are a primitive roles and C is a primitive concept
    \item \conceptdistanceelement{C}{R}{D} where R is restricted to complexity at most $2$
\end{itemize}

\bibliographystyle{plain}
\bibliography{literature.bib}

\end{document}
